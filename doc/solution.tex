\documentclass{article}
\usepackage[utf8]{ctex} % For Chinese characters, if needed
\usepackage{listings}
\usepackage{geometry}
\usepackage{fancyhdr}
\usepackage{float}

\geometry{a4paper, left=2.5cm, right=2.5cm, top=2.5cm, bottom=2.5cm}

\pagestyle{fancy}
\fancyhf{}
\chead{Data Lab 实验报告}
\cfoot{\thepage}

\lstset{
  language=C,
  breaklines=true,
  basicstyle=\ttfamily\small,
  keywordstyle=\bfseries,
  commentstyle=\itshape,
  showstringspaces=false,
  numbers=left,
  numberstyle=\tiny,
  frame=single,
  captionpos=b
}

\title{Data Lab 实验报告}
\author{<你的名字>}
\date{\today}

\begin{document}

\maketitle

\section{实验题目与解答}

\subsection{bitXor}
\subsubsection{题目要求}
使用 \texttt{\~{}} 和 \texttt{\&} 实现 \texttt{x\^{}y}。
\begin{itemize}
    \item Legal ops: \texttt{\~{}} \texttt{\&}
    \item Max ops: 14
\end{itemize}

\subsubsection{实现代码}
\begin{lstlisting}[caption={bitXor}]
int bitXor(int x, int y)
{
  int a = ~x & y;
  int b = x & ~y;
  return ~(~a & ~b);
}
\end{lstlisting}

\subsubsection{解题思路}
\begin{itemize}
  \item 异或运算的逻辑可以表示为 $(x\wedge \neg y) \vee (\neg x \wedge y)$。
  \item 根据德摩根定律,$A \vee B$等价于$\neg (\neg A \wedge \neg B)$。
  \item 因此,我们可以将 \texttt{x\^{}y} 转换为 \texttt{\~{}((\~{}(x\& \~{}y)) \& (\~{}(\~{}x\&y)))}。即可得到代码。
\end{itemize}

\subsection{tmin}
\subsubsection{题目要求}
返回最小的二进制补码整数。
\begin{itemize}
    \item Legal ops: \texttt{!} \texttt{\~{}} \texttt{\&} \texttt{\^{}} \texttt{|} \texttt{+} \texttt{<<} \texttt{>>}
    \item Max ops: 4
\end{itemize}

\subsubsection{实现代码}
\begin{lstlisting}[caption={tmin}]
int tmin(void)
{
  return 1 << 31;
}
\end{lstlisting}

\subsubsection{解题思路}
32位二进制补码的最小整数是 \texttt{0x80000000},即符号位为1,其余位全为0。这可以通过将整数1向左移动31位得到。

\subsection{isTmax}
\subsubsection{题目要求}
判断x是否为最大的二进制补码数。
\begin{itemize}
    \item Legal ops: \texttt{!} \texttt{\~{}} \texttt{\&} \texttt{\^{}} \texttt{|} \texttt{+}
    \item Max ops: 10
\end{itemize}

\subsubsection{实现代码}
\begin{lstlisting}[caption={isTmax}]
int isTmax(int x)
{
  int y = x + 1;
  int is_neg1 = !!y;
  int is_tmax_or_neg1 = !(~(x ^ y));
  return is_tmax_or_neg1 & is_neg1;
}
\end{lstlisting}

\subsubsection{解题思路}
\begin{itemize}
  \item 最大的二进制补码数 \texttt{Tmax} 是 \texttt{0x7FFFFFFF}。观察可得到, \texttt{Tmax} 加1时,会发生溢出,结果变为 \texttt{Tmin} (\texttt{0x80000000})。
  \item 此时,\texttt{Tmin}恰好是 \texttt{\~Tmax}。所以 \texttt{x+1} 应该等于 \texttt{\~x}。
  \item 同时,\texttt{-1} (\texttt{0xFFFFFFFF}) 加1后等于0,此时 \texttt{x+1} 也等于 \texttt{\~x}。因此,我们需要排除 \texttt{x} 为-1的情况。当 \texttt{x} 为-1时,\texttt{x+1} 为0,所以可以通过检查 \texttt{x+1} 是否为0来排除这种情况。
  \item 综合以上逻辑,即可得到代码。
\end{itemize}

\subsection{allOddBits}
\subsubsection{题目要求}
判断一个数的所有奇数位是否都为1。
\begin{itemize}
    \item Legal ops: \texttt{!} \texttt{\~{}} \texttt{\&} \texttt{\^{}} \texttt{|} \texttt{+} \texttt{<<} \texttt{>>}
    \item Max ops: 12
\end{itemize}

\subsubsection{实现代码}
\begin{lstlisting}[caption={allOddBits}]
int allOddBits(int x)
{
  int mask = 0x55 + (0x55 << 8);
  mask = mask + (mask << 16);
  return !(~(x | mask));
}
\end{lstlisting}

\subsubsection{解题思路}
\begin{itemize}
  \item 首先构造一个所有奇数位为1,偶数位为0的掩码 \texttt{0xAAAAAAAA}。由于不能使用大常数,我们通过 \texttt{0x55} (二进制 \texttt{01010101}) 进行移位和加法来构造。
  \item 如果 \texttt{x} 的所有奇数位都为1,那么 \texttt{x | mask} 的结果应该是全1 (\texttt{0xFFFFFFFF})。判断即可得到代码。
\end{itemize}

\subsection{negate}
\subsubsection{题目要求}
返回 \texttt{-x}。
\begin{itemize}
    \item Legal ops: \texttt{!} \texttt{\~{}} \texttt{\&} \texttt{\^{}} \texttt{|} \texttt{+} \texttt{<<} \texttt{>>}
    \item Max ops: 5
\end{itemize}

\subsubsection{实现代码}
\begin{lstlisting}[caption={negate}]
int negate(int x)
{
  return ~x + 1;
}
\end{lstlisting}

\subsubsection{解题思路}
根据二进制补码的定义,一个数的相反数等于它的按位取反再加1。

\subsection{isAsciiDigit}
\subsubsection{题目要求}
判断x是否为ASCII码中的'0'到'9'。
\begin{itemize}
    \item Legal ops: \texttt{!} \texttt{\~{}} \texttt{\&} \texttt{\^{}} \texttt{|} \texttt{+} \texttt{<<} \texttt{>>}
    \item Max ops: 15
\end{itemize}

\subsubsection{实现代码}
\begin{lstlisting}[caption={isAsciiDigit}]
int isAsciiDigit(int x)
{
  int y = x + (~0x30 + 1); // y = x - 0x30
  int z = x + (~0x3A + 1); // z = x - 0x3A
  int is_nonneg_y = !(y >> 31);
  int is_nonpos_z = !(~(z >> 31));
  return is_nonneg_y & is_nonpos_z;
}
\end{lstlisting}

\subsubsection{解题思路}
\begin{itemize}
  \item 我们需要判断一个整数 \texttt{x} 是否在0x30到0x39之间(即ASCII码中的'0'到'9')。
  \item 将原始条件转化为更易于位运算实现的条件:判断\texttt{x - 0x30 >= 0}以及\texttt{x - 0x39 <= 0}。
  \item 等价于我们需要实现两个运算符:判断是否非负(非正)以及减法运算符。
  \item 利用上一问,可以将减法转化为和相反数的加法,对判断非负,只需判断符号位即可。
  \item 综合以上逻辑,即可得到代码。
\end{itemize}

\subsection{conditional}
\subsubsection{题目要求}
实现 \texttt{x ? y : z}。
\begin{itemize}
    \item Legal ops: \texttt{!} \texttt{\~{}} \texttt{\&} \texttt{\^{}} \texttt{|} \texttt{+} \texttt{<<} \texttt{>>}
    \item Max ops: 16
\end{itemize}

\subsubsection{实现代码}
\begin{lstlisting}[caption={conditional}]
int conditional(int x, int y, int z)
{
  int is_x_nonzero = !x;
  int mask = is_x_nonzero << 31 >> 31;
  return (y & ~mask) | (z & mask);
}
\end{lstlisting}

\subsubsection{解题思路}
\begin{itemize}
  \item 使用通用的表达式: \texttt{(y \& \~{}mask) | (z \& mask)}。我们需要根据 \texttt{x} 的值选择 \texttt{y} 或 \texttt{z}。
  \item \begin{itemize}
    \item 如果 \texttt{x} 不为0,我们需要一个全1的掩码。
    \item 如果 \texttt{x} 为0,我们需要一个全0的掩码。
  \end{itemize}
  \item 通过将 \texttt{!x} 左移31位再右移31位,就可以得到所需的掩码。
\end{itemize}

\subsection{isLessOrEqual}
\subsubsection{题目要求}
判断 \texttt{x <= y}。
\begin{itemize}
    \item Legal ops: \texttt{!} \texttt{\~{}} \texttt{\&} \texttt{\^{}} \texttt{|} \texttt{+} \texttt{<<} \texttt{>>}
    \item Max ops: 24
\end{itemize}

\subsubsection{实现代码}
\begin{lstlisting}[caption={isLessOrEqual}]
int isLessOrEqual(int x, int y)
{
  int diff_y_minus_x = y + (~x + 1);
  int is_same_sign = ((x ^ y) >> 31);
  int is_y_neg = !(y >> 31);
  return (!is_same_sign & !(diff_y_minus_x >> 31)) | (is_same_sign & is_y_neg);
}
\end{lstlisting}

\subsubsection{解题思路}
判断 \texttt{x <= y} 等价于判断 \texttt{y - x >= 0}。但是直接计算 \texttt{y - x} 会有溢出的问题。
\begin{itemize}
    \item 当 \texttt{x} 和 \texttt{y} 符号相同时,\texttt{y - x} 不会溢出。此时只需判断 \texttt{y - x} 的符号位。
    \item 当 \texttt{x} 和 \texttt{y} 符号不同时,如果 \texttt{x} 为正 \texttt{y} 为负,则 \texttt{x > y};如果 \texttt{x} 为负 \texttt{y} 为正,则 \texttt{x < y}。
\end{itemize}
代码逻辑:
\begin{enumerate}
    \item \texttt{is\_same\_sign}:通过异或两数的符号位得到。如果符号相同,结果为0;不同则为-1。
    \item 如果符号相同 (\texttt{!is\_same\_sign} 为1),则判断 \texttt{y-x} 是否非负。
    \item 如果符号不同 (\texttt{is\_same\_sign} 为-1),则判断 \texttt{y} 是否为负。如果 \texttt{y} 为负,则 \texttt{x > y},应当返回0。反之则返回1。
\end{enumerate}

\subsection{logicalNeg}
\subsubsection{题目要求}
不使用 \texttt{!} 实现 \texttt{!} 操作。
\begin{itemize}
    \item Legal ops: \texttt{\~{}} \texttt{\&} \texttt{\^{}} \texttt{|} \texttt{+} \texttt{<<} \texttt{>>}
    \item Max ops: 12
\end{itemize}

\subsubsection{实现代码}
\begin{lstlisting}[caption={logicalNeg}]
int logicalNeg(int x)
{
  int is_neg = (x >> 31) & 1;
  int is_pos = (x + (~(1 << 31))) >> 31 & 1;
  return (is_neg | is_pos) ^ 1;
}
\end{lstlisting}

\subsubsection{解题思路}
\begin{itemize}
    \item \texttt{is\_neg}: 判断 \texttt{x} 是否为负数。
    \item \texttt{is\_pos}: 判断 \texttt{x} 是否为正数。这里通过 \texttt{x + Tmin} 来判断。如果 \texttt{x} 是正数,\texttt{x+Tmin} 结果为负。
    \item 如果 \texttt{x} 是非0数,则 \texttt{is\_neg | is\_pos} 为1。如果 \texttt{x} 是0,则为0。最后异或1得到结果。
\end{itemize}

\subsection{howManyBits}
\subsubsection{题目要求}
返回表示x所需的最小二进制补码位数。
\begin{itemize}
    \item Legal ops: \texttt{!} \texttt{\~{}} \texttt{\&} \texttt{\^{}} \texttt{|} \texttt{+} \texttt{<<} \texttt{>>}
    \item Max ops: 90
\end{itemize}

\subsubsection{实现代码}
\begin{lstlisting}[caption={howManyBits}]
int howManyBits(int x)
{
  // clarifies
  int mask;
  int high_16_bits, low_16_bits, is_high_16_bits_nonzero;
  int high_8_bits, low_8_bits, is_high_8_bits_nonzero;
  int high_4_bits, low_4_bits, is_high_4_bits_nonzero;
  int high_2_bits, low_2_bits, is_high_2_bits_nonzero;
  int high_1_bit, low_1_bit, is_high_1_bit_nonzero;

  int bits = 1; // at least need 1 bit for 0
  int origin_x = x;
  // if x is negative, convert x to positive
  int is_neg = ((x >> 31) & 1) << 31 >> 31; // all 1s if negative, all 0s if positive
  x = (is_neg & ~x) | (~is_neg & x);        // if x is neg, x = ~x
  // binary search for the highest 1 bit
  // search in [16,31]
  high_16_bits = x >> 16;
  low_16_bits = x ^ (high_16_bits << 16);
  is_high_16_bits_nonzero = !!high_16_bits;   // 1 for nonzero, 0 for zero
  mask = is_high_16_bits_nonzero << 31 >> 31; // all 1s if nonzero, all 0s if zero
  bits = bits + (is_high_16_bits_nonzero << 4);
  x = (mask & high_16_bits) | (~mask & low_16_bits); // if high 16 bits is nonzero, x = high 16 bits else low 16 bits
  // search in [8,15]
  high_8_bits = x >> 8;
  low_8_bits = x ^ (high_8_bits << 8);
  is_high_8_bits_nonzero = !!high_8_bits; // 1 for nonzero, 0 for zero
  mask = is_high_8_bits_nonzero << 31 >> 31;
  bits = bits + (is_high_8_bits_nonzero << 3);
  x = (mask & high_8_bits) | (~mask & low_8_bits); // if high 8 bits is nonzero, x = high 8 bits else low 8 bits
  // search in [4,7]
  high_4_bits = x >> 4;
  low_4_bits = x ^ (high_4_bits << 4);
  is_high_4_bits_nonzero = !!high_4_bits; // 1 for nonzero, 0 for zero
  mask = is_high_4_bits_nonzero << 31 >> 31;
  bits = bits + (is_high_4_bits_nonzero << 2);
  x = (mask & high_4_bits) | (~mask & low_4_bits); // if high 4 bits is nonzero, x = high 4 bits else low 4 bits
  // search in [2,3]
  high_2_bits = x >> 2;
  low_2_bits = x ^ (high_2_bits << 2);
  is_high_2_bits_nonzero = !!high_2_bits; // 1 for nonzero, 0 for zero
  mask = is_high_2_bits_nonzero << 31 >> 31;
  bits = bits + (is_high_2_bits_nonzero << 1);
  x = (mask & high_2_bits) | (~mask & low_2_bits); // if high 2 bits is nonzero, x = high 2 bits else low 2 bits
  // search in [1,1]
  high_1_bit = x >> 1;
  low_1_bit = x ^ (high_1_bit << 1);
  is_high_1_bit_nonzero = !!high_1_bit; // 1 for nonzero, 0 for zero
  mask = is_high_1_bit_nonzero << 31 >> 31;
  bits = bits + is_high_1_bit_nonzero;
  x = (mask & high_1_bit) | (~mask & low_1_bit); // if high 1 bit is nonzero, x = high 1 bit else low 1 bit
  // if x is not 0 or -1, need one more bit for sign
  bits = bits + ((!!(origin_x ^ (origin_x >> 31))) & 1);
  return bits;
}
\end{lstlisting}

\subsubsection{解题思路}
\begin{itemize}
    \item 这个问题实际上是,对正数,可以删去符号位后的前导0;对负数,可以删去符号位后的前导1。然后计算剩余位数,再加上符号位。
    \item 先对任何的负数,将其转换为正数(按位取反)。这样正负数都可以统一处理。
    \item 然后,我们需要找到最高的1位所在的位置。可以通过二分查找的方法,逐步缩小范围。具体步骤如下:
      \begin{enumerate}
        \item 检查高16位是否有1。如果有,说明最高1位在高16位,更新范围为高16位,并将结果加16。
        \item 否则,最高1位在低16位,更新范围为低16位。
        \item 重复上述过程,依次检查高8位、高4位、高2位和高1位。
        \item 每次找到最高1位所在的范围后,都将结果加上相应的位数。
      \end{enumerate}
    \item 最后,加上符号位,并处理特殊情况(0和-1),即可得到正确的代码。
\end{itemize}

\subsection{floatScale2}
\subsubsection{题目要求}
返回浮点数f的2倍。
\begin{itemize}
    \item Legal ops: Any integer/unsigned operations incl. \texttt{||}, \texttt{\&\&}. also \texttt{if}, \texttt{while}
    \item Max ops: 30
\end{itemize}

\subsubsection{实现代码}
\begin{lstlisting}[caption={floatScale2}]
unsigned floatScale2(unsigned uf)
{
  unsigned exp = uf << 1 >> 24 << 23;
  unsigned frac = uf & 0x007FFFFF;
  unsigned sign = uf >> 31 << 31;
  // NaN or infinity
  if (exp == 0x7F800000)
    return uf;
  // denorm
  if (exp == 0)
  {
    // frac overflow
    if (frac >> 22)
      return ((frac << 1) & 0x007FFFFF) + sign + 0x00800000;
    else
      return sign + (frac << 1);
  }
  // norm
  else
  {
    // exp overflow
    if (exp == 0x7F800000)
      return uf >> 23 << 23; // return infinity
    else
      return sign + (((exp >> 23) + 1) << 23) + frac;
  }
}
\end{lstlisting}

\subsubsection{解题思路}
将浮点数乘以2,需要分情况讨论:
\begin{enumerate}
    \item \textbf{NaN或无穷大}: 指数位全为1。直接返回原值。
    \item \textbf{非规格化数}: 指数位全为0。乘以2相当于将尾数左移一位。符号位不变。如果发生尾数溢出(最高位变为1),则需要将其转换为规格化数,指数位设为1,尾数清除最高位。
    \item \textbf{规格化数}: 指数位不全为0也不全为1。乘以2相当于将指数位加1。如果指数加1后变为全1,则结果为无穷大。
\end{enumerate}

\subsection{floatFloat2Int}
\subsubsection{题目要求}
将浮点数f转换为整数。
\begin{itemize}
    \item Legal ops: Any integer/unsigned operations incl. \texttt{||}, \texttt{\&\&}. also \texttt{if}, \texttt{while}
    \item Max ops: 30
\end{itemize}

\subsubsection{实现代码}
\begin{lstlisting}[caption={floatFloat2Int}]
int floatFloat2Int(unsigned uf)
{
  unsigned exp = uf << 1 >> 24;
  unsigned frac = uf & 0x007FFFFF;
  unsigned sign = uf >> 31 << 31;
  // NaN or infinity
  if (exp == 0x7F800000)
    return 0x80000000u;
  // denorm
  if (exp == 0)
  {
    return 0;
  }
  // norm
  else
  {
    int e = exp - 127;
    if (e < 0)
      return 0;
    else if (e > 31)
      return 0x80000000u;
    else
    {
      // transform frac to 1.frac
      frac = frac + 0x00800000;
      if (e > 23)
      {
        frac = frac << (e - 23);
      }
      else
      {
        frac = frac >> (23 - e);
      }
      return sign ? -frac : frac;
    }
  }
}
\end{lstlisting}

\subsubsection{解题思路}
\begin{enumerate}
    \item \textbf{提取符号、阶码和尾数}。
    \item \textbf{处理特殊值}: 如果是NaN或无穷大,或者转换后会溢出32位整数范围,返回 \texttt{0x80000000u}。
    \item \textbf{处理非规格化数}: 非规格化数都是很小的数,转换为整数结果为0。
    \item \textbf{处理规格化数}:\begin{enumerate}
      \item \textbf{还原尾数}: 将隐藏的1加上,即 \texttt{frac | 0x00800000}。
      \item \textbf{移位}: 根据阶值 \texttt{E} 对尾数进行移位。如果 \texttt{E > 23},左移;如果 \texttt{E < 23},右移。
      \item \textbf{应用符号}: 根据符号位决定返回正数还是负数。
    \end{enumerate}
\end{enumerate}

\subsection{floatPower2}
\subsubsection{题目要求}
计算 \texttt{2.0\^{}x}。
\begin{itemize}
    \item Legal ops: Any integer/unsigned operations incl. \texttt{||}, \texttt{\&\&}. Also \texttt{if}, \texttt{while}
    \item Max ops: 30
\end{itemize}

\subsubsection{实现代码}
\begin{lstlisting}[caption={floatPower2}]
unsigned floatPower2(int x)
{
  // overflow
  if (x > 127)
    return 0x7F800000;
  // underflow
  else if (x < -126)
    return 0;
  // norm
  else
  {
    return (x + 127) << 23;
  }
}
\end{lstlisting}

\subsubsection{解题思路}
我们需要构造一个表示 \texttt{2.0\^{}x} 的浮点数。尾数部分总是1.0,所以我们只需要设置阶码。
\begin{enumerate}
    \item \textbf{溢出}: 如果 \texttt{x > 127},超过了规格化数能表示的最大指数,返回正无穷。
    \item \textbf{下溢}: 如果 \texttt{x} 太小,导致无法用非规格化数表示,返回0。
    \item \textbf{规格化数}: 当 \texttt{-126 <= x <= 127} 时,结果是规格化数。阶码 \texttt{exp = x + 127},尾数部分为0。
\end{enumerate}

\section{评测结果}
运行脚本,得到评测结果如下:
\begin{table}[H]
\centering
\begin{tabular}{|c|c|c|c|c|c|}
\hline
\multicolumn{3}{|c|}{Correctness Results} & \multicolumn{3}{c|}{Perf Results} \\
\hline
Points & Rating & Errors & Points & Ops & Puzzle \\
\hline
1 & 1 & 0 & 3 & 8 & bitXor \\
1 & 1 & 0 & 3 & 1 & tmin \\
1 & 1 & 0 & 3 & 7 & isTmax \\
2 & 2 & 0 & 3 & 7 & allOddBits \\
2 & 2 & 0 & 3 & 2 & negate \\
3 & 3 & 0 & 3 & 12 & isAsciiDigit \\
3 & 3 & 0 & 3 & 7 & conditional \\
3 & 3 & 0 & 3 & 13 & isLessOrEqual \\
4 & 4 & 0 & 3 & 9 & logicalNeg \\
4 & 4 & 0 & 3 & 79 & howManyBits \\
4 & 4 & 0 & 3 & 23 & floatScale2 \\
4 & 4 & 0 & 3 & 17 & floatFloat2Int \\
4 & 4 & 0 & 3 & 5 & floatPower2 \\
\hline
\end{tabular}
\caption{实验各题评测结果}
\end{table}
比对后可知,所有的操作数均满足题目所给的限制,且所有题目均通过了正确性测试。

\end{document}
